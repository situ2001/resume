%-------------------------------------------------------------------------------
%	SECTION TITLE
%-------------------------------------------------------------------------------
\cvsection{项目经历/开源经历}


%-------------------------------------------------------------------------------
%	CONTENT
%-------------------------------------------------------------------------------
\begin{cventries}

    %---------------------------------------------------------

    \cventry
    {\href{https://github.com/logseq/logseq}{https://github.com/logseq/logseq} 是一个\textbf{开源}的,支持块级双向链接的知识管理应用} % Subtitle
    {Logseq 活跃贡献者} % Title   
    {远程} % Location
    {2022.12 - 现在} % Date(s)
    {
        \begin{cvitems} % Description(s) of tasks/responsibilities
            \item {该项目主要使用\textbf{React}, ClojureScript, JavaScript, Tailwind CSS, DataScript, Electron开发。}
            \item {作为该应用的高频使用者,反馈并修复众多bugs,实现若干功能,为项目编写测试。主要涉及\textbf{编辑器/Electron/UX}}
            \item {成为该项目的\textbf{collaborator}。提升了该应用的\textbf{用户体验}。进一步了解\textbf{函数式编程}在实际项目中的应用。}
        \end{cvitems}
    }

    %---------------------------------------------------------

    \cventry
    {\href{https://github.com/opensumi/core}{https://github.com/opensumi/core}\ 是\textbf{开源}的、阿里与蚂蚁开发的帮助开发者快速搭建本地和云端 IDE 的框架。} % subtitle   
    {OpenSumi协同编辑模块} % title
    {远程} % Location
    {2022.07 - 2022.08} % Date(s)
    {
        \begin{cvitems} % Description(s) of tasks/responsibilities
            \item {2022年\textbf{阿里巴巴开源之夏}课题。本次开发的模块,大致实现基本的编辑器\textbf{协同编辑}功能。}
            \item {了解到如 vscode 等编辑器 / \textbf{IDE} 所需的基本架构设计和底层技术,如依赖注入、websocket、IPC/RPC等。}
            \item {了解到完整项目中除开发的其他工作,如 CI/CD、单元测试、编码/commit 规范、团队合作沟通等。}
            \item {成为该社区的\textbf{member}。第一次接触\textbf{大型前端项目}。进一步了解\textbf{面向对象编程}在实际项目中的应用。}
        \end{cvitems}
    }

    %---------------------------------------------------------

    \cventry
    {有写博客和分享知识的习惯。地址: \href{https://situ2001.com}{https://situ2001.com}} % Subtitle
    {Situ Note 个人站点} % Title   
    {} % Location
    {2022.11 - 现在} % Date(s)
    {
        \begin{cvitems} % Description(s) of tasks/responsibilities
            \item {使用 \textbf{Next.js} 与 \textbf{React} 开发。样式部分使用了 Tailwind CSS 与 DaisyUI 库。}
            \item {使用 \textbf{webhooks} 实现博客文章更新自动触发构建的流程。编写 JS 脚本,将 Markdown 文章处理后更新至远程服务器数据库。}
        \end{cvitems}
    }

    %---------------------------------------------------------
\end{cventries}
