%-------------------------------------------------------------------------------
%	SECTION TITLE
%-------------------------------------------------------------------------------
\cvsection{实习经历}


%-------------------------------------------------------------------------------
%	CONTENT
%-------------------------------------------------------------------------------
\begin{cventries}
    %---------------------------------------------------------
    \cventry
    {前端开发实习生} % Job title
    {腾讯音乐\hspace{2mm}QQ音乐业务线} % Organization
    {深圳, 中国} % Location
    {2023.05 - 现在} % Date(s)
    {
        \begin{cvitems} % Description(s) of tasks/responsibilities
            \item {参与QQ音乐移动端直播业务的web页面开发与维护 (\textbf{React})}
            \item {参与互动直播\textbf{中台基础库}的开发与维护,该基础库封装了通用函数、组件库、网络请求、性能上报、JSBridge等功能}
            \item {参与全民K歌小游戏\textbf{SDK}与\textbf{小游戏运行时容器}Atum(Android客户端)的开发与联调}
            \item {为上述项目做了若干\textbf{工程化}建设,优化并改进了项目的\textbf{开发体验},极大地提高部门的开发效率}
            \item {如书写\textbf{Node.js} CLI,将SDK开发繁杂的流程自动化}
            \item {实习期间不受限于前端实习生身份,积极主动地与同事沟通交流,获得了部门同事的认可}
        \end{cvitems}
    }

    %---------------------------------------------------------
    \cventry
    {前端开发实习生} % Job title
    {滴滴出行\hspace{2mm}网约车部门} % Organization
    {北京, 中国} % Location
    {2023.03 - 2023.05} % Date(s)
    {
        \begin{cvitems} % Description(s) of tasks/responsibilities
            \item {参与滴滴司机部落APP的\textbf{跨端页面}开发与维护,涉及 \textbf{Vue}, Thanos(Weex), JSBridge }
            \item {经历了技术方案编写与评审、开发、测试、上线、监控的\textbf{研发流程}}
        \end{cvitems}
    }
    %---------------------------------------------------------
\end{cventries}
