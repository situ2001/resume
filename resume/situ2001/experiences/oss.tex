%-------------------------------------------------------------------------------
%	SECTION TITLE
%-------------------------------------------------------------------------------
\cvsection{开源经历}


%-------------------------------------------------------------------------------
%	CONTENT
%-------------------------------------------------------------------------------
\begin{cventries}

%---------------------------------------------------------

\cventry
    {opensumi是阿里与蚂蚁开发的帮助开发者快速搭建本地和云端 IDE 的框架。\href{https://github.com/opensumi/core}{https://github.com/opensumi/core}} % subtitle   
    {OpenSumi协同编辑模块} % title
    {} % Location
    {2022.07 - 2022.08} % Date(s)
    {
      \begin{cvitems} % Description(s) of tasks/responsibilities
        \item {2022年\textbf{阿里巴巴编程之夏}课题}
        \item {本次开发的模块,大致实现基本的编辑器协同编辑功能。}
        % \item {该项目主要使用React, Mobx, Node.js进行开发。}
        \item {了解到如 vscode 等编辑器 / IDE 所需的底层技术和基本架构设计,如依赖注入、websocket、IPC等。}
        \item {了解到完整项目中除开发的其他工作,如 CI/CD、单元测试、编码/commit 规范、团队合作沟通等。}
        \item {成为该社区的member,并进一步了解了面向对象编程在实际项目中的应用。}
      \end{cvitems}
    }

%---------------------------------------------------------

\cventry
    {Logseq是一个开源的,支持块级双向链接的知识管理应用。\href{https://github.com/logseq/logseq}{https://github.com/logseq/logseq}} % Subtitle
    {Logseq 活跃贡献者} % Title   
    {} % Location
    {2022.12 - 现在} % Date(s)
    {
      \begin{cvitems} % Description(s) of tasks/responsibilities
        \item {该项目主要使用React, Tailwind CSS, ClojureScript, JavaScript, DataScript, Electron开发。}
        \item {作为该应用的日常用户,不甘于现有体验。反馈众多bug并对其进行修复。主要涉及编辑器,Electron等。}
        \item {提出并实现若干功能,如支持多种格式粘贴,便利的剪贴板数据查看工具。}
        \item {为其书写E2E测试。闲时负责维护该项目的issues区。}
        \item {成为该项目的collaborator,提升了该应用的用户体验,并进一步了解了函数式编程在实际项目中的应用。}
      \end{cvitems}
    }

%---------------------------------------------------------
\end{cventries}
