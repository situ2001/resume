%-------------------------------------------------------------------------------
%	SECTION TITLE
%-------------------------------------------------------------------------------
\cvsection{项目经历}

%-------------------------------------------------------------------------------
%	CONTENT
%-------------------------------------------------------------------------------
\begin{cventries}


    %---------------------------------------------------------

    \cventry
    {个人博客 \href{https://situ2001.com}{https://situ2001.com}} % Subtitle
    {Situ Note} % Title   
    {} % Location
    {2022.11 - 现在} % Date(s)
    {
        \begin{cvitems} % Description(s) of tasks/responsibilities
            \item {有\textbf{写博客的习惯},大部分博客发布于此。}
            \item {使用 Next.js 与 \textbf{React} 开发。样式部分使用了 Tailwind CSS 与 DaisyUI 库。}
            \item {使用 webhooks 实现博客文章更新自动触发构建的流程。编写 JS 脚本,将 Markdown 文章处理后更新至远程服务器数据库。}
            \item {使用 \textbf{serverless} 函数,配合 ORM 库 Prisma 实现了简易的访客计数等功能。}
        \end{cvitems}
    }

    %---------------------------------------------------------
\end{cventries}
